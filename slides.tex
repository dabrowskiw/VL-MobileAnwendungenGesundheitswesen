\documentclass[aspectratio=169,t]{beamer}
\usepackage[utf8]{inputenc}
\usepackage[T1]{fontenc}


\title{Mobile Anwendungen im Gesundheitswesen}
\date{WS 2019/2020}
\author[PWD]{Dr.-Ing. Piotr Wojciech Dabrowski}
\titlegraphic{Bilder/logo.png}
% https://pxhere.com/en/photo/1441931

\usepackage{HTWBeamerTemplate/beamerthemeHTW}
\subtitle{0: Allgemeines}
\addbibresource{Bilder/imagesources.bib}
\begin{document}

\setbeamertemplate{footline}[first]


\begin{frame}[noframenumbering]
    \titlepage
    \begin{textblock}{10}(4.75,15)
        \cite{logo}
    \end{textblock}
\end{frame}

\setbeamertemplate{footline}[presentationbody] 

\begin{frame}{Vorstellung}
    \note<4->{
        Erstes Mal Vorlesung, Background nicht in mobilen Applikationen\\Bitte um Geduld, Hinweise
    }
    \begin{itemize}
        \item<2-> Kurz zu mir
        \only<2-4>{
            \begin{itemize}
                \item<2-4> Kontakt: Piotr.Dabrowski@htw-berlin.de - gerne nutzen!
                \item<2-4> Repository: https://github.com/dabrowskiw/
                \item<3-4> Geboren 1981 in Warschau
                \item<3-4> Studium der Biotechnologie \& Informatik an der TU Berlin
                \item<3-4> Promotion über Auswertung von Hochdurchsatzdaten für Virus-Diagnostik
                \item<3-4> Aufbau der bioinformatischen Analytik für das NGS-Labor des RKI
                \item<3-4> Aufbau der Bioinformatics Core Facility am RKI
                \item<4> Hang zu unkonventionellen Vorlesungsmethoden - Feedback erwünscht!
            \end{itemize}
        }
        \item<5-> Der Todesstern \& (ausgewählte) andere Hilfsmittel
        \item<9-> Sie \& Ihre Vorstellungen, Motivation und Vorkenntnisse
        \only<10->{
            \begin{itemize}
                \item Kommt gleich!
            \end{itemize}
        }
    \end{itemize}
    \only<5-8>{
        \begin{textblock}{15}(2,7)
            \includegraphics[width=3cm]{Bilder/Todesstern.jpg}
        \end{textblock}
    }
    \only<6-8>{
        \begin{textblock}{15}(5.5,8.25)
            \includegraphics[width=3cm]{Bilder/Tischnamensschild.png}
        \end{textblock}
    }
    \only<7-8>{
        \begin{textblock}{15}(9,7.5)
            \includegraphics[angle=77,origin=c,height=2cm]{Bilder/Tischnamensschild.png}
        \end{textblock}
    }
    \only<8>{
        \begin{textblock}{15}(11.5,8)
            \includegraphics[width=3cm]{Bilder/Whiteboard.png}
        \end{textblock}
    }
\end{frame}

\begin{frame}{Fragen der Vorlesung}
    \note{
        \begin{itemize}
            \item Entwicklung: Besonders wichtig bei Gesundheitswesen: Stabilität, einfache Anwendbarkeit (patient compliance)
            \item Was sind personenbezogene Daten, wann darf man sie verarbeiten - und wann sollte man sie verarbeiten etc.
        \end{itemize}
    }
    \begin{itemize}
        \item Warum sind mobile Applikationen im Gesundheitswesen sinnvoll? Wo werden sie bereits eingesetzt?
        \item<2-> Worauf muss man bei der Entwicklung mobiler Applikationen für das Gesundheitswesen achten?
        \item<3-> Was sind Herausforderungen im Zusammenhang mit der Verarbeitung potentiell sensitiver Daten?
    \end{itemize} 
\end{frame}

\begin{frame}{Inhalt der Übung}
    Entwicklung einer mobilen medizinischen Applikation.
    \begin{itemize}
        \item<2-> Zusammenfinden in 4 möglichst balancierten Teams (5-6 Personen)
        \item<3-> Entwickeln einer Produktidee
        \item<4-> Erstellung von Strategie, Mockups etc.
        \item<5-> Implementation eines Proof of Concept
        \item<6-> Erstellung einer Produktpräsentation (25 Minuten/Gruppe)
    \end{itemize}
    \only<7->{Credit to: Thorsten Knape.}
\end{frame}

\begin{frame}{Benotung}
    \note{
        \begin{itemize}
            \item PoC nicht ``fertig werden'', Idee und Herangehensweise ausschlaggebend.
            \item Produktpräsentation:
            \begin{itemize}
                \item Ideen, Herausforderungen, Umgehen mit Herausforderungen so darstellen, dass sie für die Anderen nachvollziehbar und nützlich sind!
                \item Vorstellung des PoC: Verkaufsevent, warum sind wir besser als alle anderen!
            \end{itemize}
        \end{itemize} 
    }
    \begin{itemize}
        \item<1-> Projektaufgabe: $70\%$
        \only<2>{
            \begin{itemize}
                \item Funktionsfähiger technischer Prototyp: $20\%$
                \item Dokumentation des Codes (Vorhandensein, Übereinstimmung Dokumentation/Code): $10\%$
                \item Projektdokumentation als PDF/Wiki: $10\%$
                \item Konsistenter Code-Stil: $10\%$
                \item Vorhandensein dokumentierter Tests: $10\%$
                \item Kleinteilige Commits: $10\%$
            \end{itemize}
        }
        \item<3-> Präsentation: $30\%$
        \only<4>{
            \begin{itemize}
                \item 25 Minuten/Gruppe + Fragen
                \item \textbf{Für alle verständliche} Vorstellung wichtiger Gedanken, Herausforderungen, Entscheidungen, Lösungsansätze
                \item PoC-Präsentation mit Verkaufscharakter: Warum ist/wird das die beste App auf der Welt?
                \item Emfpehlung: Story erzählen.
            \end{itemize}
        }
        \item Abgabe Projektaufgabe (git Release mit allen Elementen) und Slides für Präsentation (Mail mit Anhang): 04.02.2019 12:15 (Eingang Mail + Release-Datum)
        \item<6-> Umgesetzte Verbesserungsvorschläge/Fehlerkorrekturen
        \only<7>{
            \begin{itemize}
                \item $2.5\%$ pro Stück
                \item Maximal 2 pro Semester
                \item Wenn als pull request: Doppelte Punktzahl
            \end{itemize}
        }
    \end{itemize}
\end{frame}

\begin{frame}{Vorläufige Zeitplanung}
    \includegraphics[width=\textwidth]{Bilder/Zeitplan.png}
\end{frame}

\stepcounter{slidesection}
\setbeamertemplate{background}[bgfirst]
\setbeamertemplate{footline}[first]
\subtitle{\theslidesection: Kurzer Einblick in das Gesundheitssystem}
\titlegraphic{Bilder/logo1.png}
\begin{frame}[noframenumbering]
\titlepage
\begin{textblock}{10}(4.75,15)
\cite{GesundheitssystemLogo}
\end{textblock}
\end{frame}
\setbeamertemplate{footline}[presentationbody] 
\setbeamertemplate{background}[bgbody]

\begin{frame}{Begriffsdefinition(en)}
    \begin{definition}
        Das Gesundheitswesen ist die Gesamtheit eines organisierten Handelns als Antwort auf das Auftreten von Krankheit und Behinderung und zur Abwehr gesundheitlicher Gefahren.
    \end{definition}
    \only<2->{
        \begin{definition}
            Das Gesundheitswesen setzt sich aus allen Instituten, Einrichtungen, Personen und allen Maßnahmen zusammen, die für die Bevölkerung gesundheitsfördernd und -erhaltend sind, vorbeugend gegen Verletzungen und Krankheit wirken sowie diese behandeln.
        \end{definition}
    }
\end{frame}

\begin{frame}{Akteure im Gesundheitssystem (Überblick)}
    \begin{figure}[h!]
        \includegraphics[height=5.5cm]{Bilder/Gesundheitssystem.pdf}
        \caption{Akteure im Gesundheitssystem. Eigene Abbildung in Anlehnung an \cite{SmartHealth}}
    \end{figure}
\end{frame}

\begin{frame}{...und ein kurzer Blick in die Tiefe}
    \begin{figure}[h!]
        \includegraphics[height=5.5cm, right]{Bilder/GesundheitssystemAkteureBund.pdf}
        \caption{Einige Interaktionen auf Bundesebene. Eigene Abbildung.}
    \end{figure}
\end{frame}

\begin{frame}{Datenaustausch (Beispiel GKV)}
    \note{
        \begin{itemize}
            \item Richtlinien für den Datenaustausch im Gesundheits- und Sozialwesen, 110 Seiten
            \item Häufig im Einsatz: Mainframe, COBOL
        \end{itemize}
    }
    \begin{figure}[h!]
        \includegraphics[height=5.5cm, right]{Bilder/DatenfluesseGesundheitssystem.pdf}
        \caption{Beispiele für Austausch personenbezogener Daten im Gesundheitssystem. Eigene Abbildung.}
    \end{figure}
\end{frame}

\begin{frame}{Herausforderungen}
    \note{
        \begin{itemize}
            \item Steigende Anzahl an Ärzten
            \item Sinkende Zeit pro Patient
            \item Gründe: 
            \begin{itemize}
                \item Demographischer Wandel
                \item Bessere Behandelbarkeit trivialer Erkrankungen führt zu mehr chronischen und komplexen Erkrankungen
                \item<4> Zunehmende Pflegebedürftigkeit
                \item<4> Steigende Dokumentationspflichten
                \item<4> Steigende Komplexität der Diagnose/Behandlung
            \end{itemize}
        \end{itemize}
    }
    \begin{minipage}{.4\textwidth}
        \begin{figure}[h!]
            \includegraphics[height=5.5cm]{Bilder/HausarztVersorgung.png}
            \caption{Versorgung mit Hausärzten in 2017 \cite{HausarztVersorgung}}
        \end{figure}
    \end{minipage}
    \only<2->{
        \begin{textblock}{10}(6,0.5)
            \begin{figure}[h!]
                \frame{\includegraphics[width=6cm]{Bilder/Rekordlebenserwartungen.PNG}}
                \caption{Anstieg der Rekordlebenserwartung und der Lebenserwartung in Deutschland \cite{GesundheitKrankheitAlter}}
            \end{figure}
        \end{textblock}
    }
    \only<3->{
        \begin{textblock}{10}(6.7,6.7)
            \begin{figure}[h!]
                \frame{\includegraphics[width=6cm]{Bilder/InanspruchnahmeAerzte.png}}
                \caption{Inanspruchnahme von Ärzten (letzte 12 Monate) nach Alter und Arzt \cite{GesundheitKrankheitAlter}}
            \end{figure}
        \end{textblock}
    }
    \only<4->{
        \begin{textblock}{10}(2.8,3.8)
            \begin{figure}[h!]
                \frame{\includegraphics[width=6.5cm]{Bilder/Einpersonenhaushalte.png}}
                \caption{Entwicklung der Haushaltsgröße \cite{GesundheitKrankheitAlter}}
            \end{figure}
    \end{textblock}
    }
\end{frame}

\begin{frame}{IT-Unterstützung}
  \note{
  \begin{itemize}
      \item UC: Z.B. Blutdruck, Blutzucker, Sturzsensor, Lokationssensor (Wandern von Alzheimer-Patienten), Schweiß- und Herzfrequenzüberwachung zur Vorhersage epileptischer Anfälle etc.
      \item Benachrichtigung in Stufen, z.B. Angehörige, dann Pflegepersonal
      \item Erinnerung an Medikamente (Problem: Adhärenz!)
      \item Generell: Ortsunabhängige Pflege (zu Hause)
      \item Therapieunterstützung: Z.B. Kommunikation bei Depression, Gefühlstagebuch
      \item <2-> Automatische Übertragung von Geräten zu zentralen Datenbanken, automatische Vorauswertungen von Bilddaten
      \item<2-> Standardisierung - großes Problem in Medizintechnik
      \item<3-> Abgleich verschriebener Medikamente mit gelber oder roter Liste
      \item<5-> Automatische Benachrichtigung bei Ablauf von Blutkonserven, RFID-Kennzeichnung von OP-Besteck um Desinfektionszyklen einzuhalten etc.
  \end{itemize}}
  \begin{itemize}
      \item Ubquitous Computing: Sensorik und automatische Benachrichtigungen, insbesondere in der Pflege
      \item Telemedizin, Therapieunterstützung
      \item<3-> Verbesserung der Technikintegration
      \item<4-> Expertensysteme zur Behandlungsunterstützung
      \item<5-> Dokumentationsunterstützung
      \item<6-> Logistik
      \item<7-> Allgemein: Entlastung und Effizienzsteigerung, um bei gleichem Personal mehr Zeit für die Patienten zu haben
  \end{itemize}
\end{frame}

\begin{frame}{Aber warum Apps?}
    \begin{columns}
        \begin{column}{0.5\textwidth}
            \only<2->{
                \vspace{-0.6cm}
                \begin{figure}
                    \includegraphics[width=0.75\textwidth]{Bilder/Concert2006.png}
                    \caption{Basement Jaxx, 2006 \cite{Concert2006}}
                \end{figure}
                \vspace{-1cm}
                \begin{figure}
                    \includegraphics[width=0.75\textwidth]{Bilder/Concert2015.png}
                    \caption{AC/DC, 2015 \cite{Concert2015}}
                \end{figure}
            }
        \end{column}
        \begin{column}{0.5\textwidth}
            \begin{itemize}
                \item<3-> Durchdringung des Berufs- und Privatlebens mit digitalen Services
                \item<3-> Mentalität: Live erleben, digital teilen - Ungeduld
                \item<4-> Steigende Verfügbarkeit und Abhängigkeit von Handys
            \end{itemize}
        \end{column}
    \end{columns}
\end{frame}

\begin{frame}{App-Beispiel}
    ``Ich selbst habe eine App auf dem Handy, die mit 20 oder 30 Fragen Diagnosen genauer trifft als viele Ärzte, weil sie auf so viele Studien und Informationen zurückgreifen kann, wie es kein Arzt alleine kann.''\\
    Wer hat das über welche App gesagt?
    \only<2->{
        \\\vspace{0.5cm}Bundesgesundheitsminister Jens Spahn, 2018 \cite{SpahnApp} über ``Ada - Deine Gesundheitshelferin''.
    }
\end{frame}

\begin{frame}{ADA health companion}
    Google-Suche:
    \begin{itemize}
        \item \textit{A Look at Ada's AI-Powered Health Companion}: Ada's AI-powered health companion helps individuals understand and manage their health. In practice, Ada works much like a GP in your pocket. \cite{ADA1}
        \item<2-> \textit{Ada is an AI-powered doctor app and telemedicine service}: In my brief testing of the app, I plugged in the symptoms of a sore or red eye. After drilling through a quite extensive set of questions, many of which appeared to relate to the answers I’d previously given, the Ada app provided three possible conditions, and advised that they could be successfully treated at home. \cite{ADA2}
        \item<3-> \textit{Health companion app Ada Health raises EUR 40 million in first funding round} \cite{ADA3}
    \end{itemize}
\end{frame}

\begin{frame}{ADA health companion II}
    Gezielte Suche nach Studien zu Zuverlässigkeit:
    \begin{itemize}
        \item \textit{Doctors warn over diagnosis apps amid Ada launch}: Both the Australian Medical Association and Royal Australian College of GPs said they were concerned about the accuracy of the Ada system, and its potential to either falsely reassure people about their health or alarm them unnecessarily. \cite{ADA4}
        \item<2-> \textit{'Trust but verify' – five approaches to ensure safe medical apps}: We eagerly look forward to a time when medical apps might be relied upon to do much more complex tasks than simply calculate formulae or illustrate inhaler technique. [...] The potential for benefit remains vast and the degree of innovation is inspiring, but it turns out we are much earlier in the maturation phase of medical apps than many of us would have liked to believe. \cite{ADA5}
    \end{itemize}
\end{frame}

\begin{frame}{Ökosystem}
    \begin{columns}
        \begin{column}{0.5\textwidth}
            \begin{figure}
                \centering
                \includegraphics[width=\textwidth]{Bilder/Iot_apps.jpg}
                \caption{IoT, Smart Cities und Smart Health \cite{IoTapps}}
            \end{figure}
        \end{column}
        \begin{column}{0.5\textwidth}
            Gesundheits-Apps leben nicht isoliert.\\\vspace{0.5cm}
            \only<2->{
                ``The  maturation  and  adoption  of  computing  technologies have  dramatically  changed  the  face  of  healthcare. [...] In  the long  term,  integrating  technology  into  city-wide healthcare can reduce costs for the city and its citizens.'' \cite{Smart}
            }
        \end{column}
    \end{columns}
\end{frame}


\begin{frame}{Rahmenbedingungen}
    \note{
        Whiteboard-Aufgabe: Wann kann eine Gesundheits-App gefährlich sein? 
        \textbf{Zeit: Max. 5 Minuten}, wenn Fertig: Namensschild auf Bildschirm.
        \\KI: Großes Interesse, 
    }
    Kann von Apps eine Gefahr ausgehen?
    \only<2->{
        Einige Beispiele:
        \begin{itemize}
            \item Folgen von Fehlfunktionen, z.B. Einnahme falscher Medikamente
            \item<2-> Fehldiagnose, zu später Arztbesuch
            \item<3-> ``29 Prozent der Deutschen sehr oder ziemlich an einem Service interessiert, der den möglichen Einfluss der aktuellen Ernährungsweise auf die künftige Gesundheit ermittelt'' \cite{InteresseKI} - zu hohes Vertrauen?
            \item<3-> Falsche Haltung/Intensität bei Übungen
            \item<4-> Abfluss persönlicher Daten: ``Gesundheits-Apps halten die datenschutzrechtlichen Anforderungen häufig nicht ein'' \cite{CHARISMABMG}
            \item<5-> Verlust von Körpergefühl, Verstärkung von Krankheitsängsten (Cyberchondrie)
        \end{itemize}
    }
    \only<6->{...also müsste es irgendeine Regelung geben?}
    \only<7->{ $\rightarrow$ Medizinproduktegesetz.}
\end{frame}

\begin{frame}{Medizinprodukte}
    \note{
        Alles, was kein Medikament ist und bei der
        \begin{itemize}
            \item Erkennung
            \item Verhütung
            \item Überwachung
            \item Behandlung
            \item Linderung
            \item Kompensierung
        \end{itemize}
        von Krankheiten, Verletzungen oder Behinderungen oder bei der Empfängnisregelung hilft (und durch Medikamente unterstützt werden kann). 
    }
    \footnotesize{
        Medizinprodukte sind alle einzeln oder miteinander verbunden verwendeten Instrumente, Apparate, Vorrichtungen, Software, Stoffe und Zubereitungen aus Stoffen oder andere Gegenstände einschließlich der vom Hersteller speziell zur Anwendung für diagnostische oder therapeutische Zwecke bestimmten und für ein einwandfreies Funktionieren des Medizinproduktes eingesetzten Software, die vom Hersteller zur Anwendung für Menschen mittels ihrer Funktionen zum Zwecke
        \begin{itemize}
            \item der Erkennung, Verhütung, Überwachung, Behandlung oder Linderung von Krankheiten,
            \item der Erkennung, Überwachung, Behandlung, Linderung oder Kompensierung von Verletzungen oder Behinderungen,
            \item der Untersuchung, der Ersetzung oder der Veränderung des anatomischen Aufbaus oder eines physiologischen Vorgangs oder
            \item der Empfängnisregelung
        \end{itemize}
        zu dienen bestimmt sind und deren bestimmungsgemäße Hauptwirkung im oder am menschlichen Körper weder durch pharmakologisch oder immunologisch wirkende Mittel noch durch Metabolismus erreicht wird, deren Wirkungsweise aber durch solche Mittel unterstützt werden kann.
    }
\end{frame}

\begin{frame}{Medizinproduktegesetz}
    \note{
        Ziel: Keine große Expertise in MPG aufbauen, das wäre eigene VL. Gefühl für die Komplexität bekommen! (heißt aber nicht ``nicht klausurrelevant''!)\\
        Wichtig zu bedenken: Ursprünglich und hauptsächlich Gedanke an Prothesen, Bandagen etc., aber auch anwendbar auf Gesundheits-Apps!
        \\
        \textbf{Risikoklasse}: Apps meist Klasse \textbf{I}, außer z.B. Verhütung/STD-Verhinderung Klasse \textbf{IIb}, Diagnose oder Kontrolle von Vitalfunktionen \textbf{IIa oder IIb}
        \\
        \textbf{ZLG}: Zentralstelle der Länder für Gesundheitsschutz bei Arzneimitteln und Medizinprodukten
    }
    Das MPG regelt den Umgang mit Medizinprodukten. Unter Anderem:
    \begin{itemize}
        \item Wann das Inverkehrbringen verboten ist
        \only<1>{
            \begin{itemize}
                \item Wenn Verdacht auf Gefährdung besteht
                \item Wenn sie Leistung versprechen, die sie nicht erbringen
                \item Sicherer Erfolg/Schadensfreiheit versprochen wird aber nicht garantiert werden kann
            \end{itemize}
        }
        \item<2-> Vorgehen beim Inverkehrbringen
        \only<2>{
            \begin{itemize}
                \item Verantwortlich: Hersteller, \textbf{Einführer} oder Bevollmächtigter
                \item Einordnung in Risikoklasse (93/42/EWG)
                \item CE-Kennzeichen notwendig (außer Klasse I durch ZLG benannte Benannte Stelle)
                \item Verständliche Anleitung in Deutscher Sprache
            \end{itemize}
        }
        \item<3-> Details zur klinischen Bewertung und Leistungsbewertung
        \only<3>{
            \begin{itemize}
                \item Klinische Bewertung anhand klinischer Daten notwendig, falls nicht andere Daten ausreichend sind (begründete Ausnahmefälle sowie in-vitro-Diagnostika)
                \item Klinische Prüfung nur mit Genehmigung der Ethikkommission, Patienteneinwilligung etc.
            \end{itemize}
        }
        \item<4-> Überwachung
        \only<4>{
            \begin{itemize}
                \item Meldepflicht für Verantwortliche
                \item Rechte der Behörden zur Überprüfung (Betretung und Besichtigung von Geschäftsräumen, Prüfung von Produkten und Unterlagen etc.)
                \item Anordnung zur Schließung des Betriebs bei Gefahr für öffentliche Gesundheit
            \end{itemize}
        }
        \item<5-> Strafen, z.B.
        \only<5>{
            \begin{itemize}
                \item Unerlaubtes Inverkehrbringen oder Betreiben: Bis 3 Jahre
                \item Unerlaubtes Anbingen von CE-Kennzeichen: Bis 1 Jahr
                \item Unerlaubtes Anwenden eines in-vitro-Diagnostikums: Bis 30'000 Euro
                \item Eine Überwachungsmaßnahme nicht zulassen: Bis 30'000 Euro
            \end{itemize}
        }
    \end{itemize}
    \only<6->{Aber: Nicht jede Gesundheitsapp ist ein Medizinprodukt! Abgrenzung: Apps für reine Sportzwecke, Fitness, Wellness oder Ernährung. Siehe auch BfArM-Orientierungshilfe \cite{BfArMOrientierungshilfe}.}
    \only<7>{\\\textbf{Wichtig}: Das sind nur die Regeln in Deutschland. Andere Länder, andere Regeln, andere Strafen!}
\end{frame}

\stepcounter{slidesection}
\setbeamertemplate{background}[bgfirst]
\setbeamertemplate{footline}[first]
\subtitle{\theslidesection: Entwicklungsaufgabe}
\titlegraphic{Bilder/logo3.jpg}
\begin{frame}[noframenumbering]
    \titlepage
    \begin{textblock}{10}(4.75,15)
        \cite{logo3}
    \end{textblock}
\end{frame}
\setbeamertemplate{footline}[presentationbody] 
\setbeamertemplate{background}[bgbody]

\begin{frame}{Startup-Gründung}
    \note{
        Idee: Vielleicht noch nicht, machen wir aber im Ideenfindungs-Workshop!
    }
    Sie haben:
    \begin{itemize}
        \item Team und Willen, ein Startup zu gründen
        \item Eine Idee für eine Gesundheits-App
        \item Kontakte zu Investoren
    \end{itemize}
    \only<2->{
        Sie brauchen:
        \begin{itemize}
            \item Einen Business-Plan
            \item Ein Proof-of-Concept
        \end{itemize}
    }
    \only<3->{...und das in 4 Monaten! Was nun?}
\end{frame}

\begin{frame}{I: Zielgruppe definieren}
    Wo gibt es:
    \begin{itemize}
        \item Viele potentielle Anwender
        \item Großen Lösungsbedarf
        \item Großes Verbesserungspotential
        \item Allgemein: Hohes Marktpotential
    \end{itemize}
\end{frame}

\begin{frame}{I: Zielgruppe definieren}
    \begin{figure}
        \centering
        \includegraphics[width=0.7\textwidth]{Bilder/Krankheitstage.jpg}
        \caption{Anteil der Krankheitstage (Gesamt 510 Millionen) AOK-Versicherter nach Diagnose. Eigene Abbildung, Daten der GBE Bund \cite{GBEKrankheitstage}}
    \end{figure}
\end{frame}

\begin{frame}{II: Exploration}
    \note{
        \begin{itemize}
            \item Alkohol: $10\%$ der Arbeitnehmer risikant
            \item Spiele: $6.5\%$ der Arbeitnehmer risikant
            \item Neubildungen: Nur $3.5\%$ Fehltage, aber teures und großes Risiko für z.B. Bauarbeiter. UV-Vorhersage-Apps
        \end{itemize}
        
    }
    \begin{itemize}
        \item Muskel-Skelett-System: Laut Ärzteblatt hat jeder 3. Erwachsene Rückenbeschwerden
        \item Depression: Laut WHO 4.6 Millionen Deutsche mit Depression
        \item Suchtprobleme (Alkohol, Spiele, Rauchen): Laut DAK-Gesundheitsreport 2019 über $10\%$
        \item Wohlstandskrankheiten, demografischer Wandel
    \end{itemize}
    \only<2->{
        Nächster Schritt: Literaturscreening zu Abhilfen.
    }
\end{frame}

\begin{frame}{Umsetzungsideen}
    Wissen, was zu tun ist, ist einfach. Es tun ist schwierig.
    \begin{itemize}
        \item Integration von Sensorik, Datenanalyse (``intelligentes'' System)
        \item Plattform-Übergreifende Lösung?
    \end{itemize}
\end{frame}

\begin{frame}{Teambildung: Vorstellung}
    \note{
        Um das Ganze ein wenig aufzulockern...
        \begin{itemize}
            \item Ziel: Entwicklung einer App im 5er-Team
            \item Manche haben mehr, manche weniger technischen Background
            \item Studierende aus IMI anwesend, wichtige Kompetenz!
            \item Informationen, um ausgewogene Teams zu erstellen
        \end{itemize}
        Zeit: 10 Minuten. \textbf{Wenn fertig, Namensschild auf Bildschirm!}
        \\\vspace{0.5cm}
        Man glaubt gar nicht, wie häufig man sich kurz vorstellen muss - guten Eindruck machen ist wichtig, also üben!
    }
    \only<2->{
        Kurzprofil:
        \begin{itemize}
            \item Name
            \item Bisherige Erfahrung (Sprachen, Projekte etc.) in:
            \begin{itemize}
                \item Programmierung
                \item App-Entwicklung
                \item (UI-)Design
            \end{itemize}
            \item Relevante Interessen, idealerweise mit erster App-Idee
        \end{itemize}
    }
    \only<3->{Vorstellung: 2 Minuten + Profilbogen}
\end{frame}

\begin{frame}{Hausaufgabe!}
    Gruppenfindung\only<2->{... oder zumindest schon mal Gedanken machen. }
    \only<3->{
        \begin{itemize}
            \item Auf ausgewogene Teams achten!
            \item Bei Teilteams: Welche Kompetenzen brauchen wir noch?
            \item Erste Ideenabstimmung zum Thema der App
        \end{itemize}
    }
    \only<4->{Nächste Woche ist noch Zeit zum Finalisieren.}
\end{frame}

%\begin{frame}{App-Ideen}
%In der Woche drauf!
%\end{frame}

\stepcounter{slidesection}
\setbeamertemplate{background}[bgfirst]
\setbeamertemplate{footline}[first]
\subtitle{\theslidesection: Technische Grundlagen}
\titlegraphic{Bilder/logo2.png}
\begin{frame}[noframenumbering]
    \titlepage
    \begin{textblock}{10}(4.75,15)
        \cite{ProgrammingLogo}
    \end{textblock}
\end{frame}
\setbeamertemplate{footline}[presentationbody] 
\setbeamertemplate{background}[bgbody]


\begin{frame}{Sourcecodeverwaltung}
    \note{
        Bevor wir anfangen, Code zu schreiben, müssen wir uns Gedanken darüber machen, wie wir das tun...
        \begin{itemize}
            \item Perforce
            \begin{itemize}
                \item Einsatz in großen Unternehmen (größtes Repo bei Google, 18/20 top-Spieleherstellern, Netflix etc.)
                \item Ändert darunterliegende Technologie nach Bedarf (aktuell git.kompatibel, aber erste Version 10 Jahre vor git)
            \end{itemize}
            \item Subversion: Tracking von Verschieben, Umbenennen etc.
            \item git: Durch Kommerzialisierung von BitKeeper
            \item mercurial: Gleiches Ziel wie git (Linux-Kernel), heutzutagez.B. bei Facebook und Mozilla im Einsatz
            \item Einer der großen heiligen Kriege: git vs. mercurial
            \item Derzeit primär im Einsatz: SVN, Perforce, TFS (zentral), git, mercurial (dezentral)
        \end{itemize}
    }
    \begin{itemize}
        \item Ursprüngliche Herangehensweise: Datei\_final\_final\_realfinal\_12.java...
        \item<2-> 1972: Bell labs, SCCS (source code control system, single user, nur Textdateien)
        \item<3-> 1986: CVS (concurrent version system, zentral, dateibasiert)
        \item<4-> 1995: Helix core (Perforce)
        \item<5-> 2000: SVN (subversion, zentral, verzeichnisbasiert)
        \item<6-> 2004: TFS (team foundation server, Microsoft, zentral/git, verzeichnisbasiert mit bug tracker etc.)
        \item<7-> 2005: git (Linus Torvalds, dezentral, verzeichnisbasiert)
        \item<8-> 2005: mercurial (dezentral, verzeichnisbasiert)
    \end{itemize}
\end{frame}

\begin{frame}{Sourcecodeverwaltung (git)}
    \note{Abgabe von Übungsaufgaben nur als Release-Link in git-Repository}    
    \begin{minipage}{.4\textwidth}
        \begin{figure}[h!]
            \frame{\includegraphics[height=5.5cm]{Bilder/gitbranching2.png}}
            \caption{Teamarbeit mit mehreren remotes \cite{gitbranching}}
        \end{figure}
    \end{minipage}
    \only<2->{
        \begin{textblock}{6}(10,0.2)
            \begin{figure}[h!]
                \includegraphics[height=7.3cm]{Bilder/gitbranching.png}
                \caption{Komplexe branching-Strategie \cite{gitbranching}}
            \end{figure}
        \end{textblock}
    }
\end{frame}

\begin{frame}{Übung: git}
    \note{
        \begin{itemize}
            \item Gitlab: Open Source (inhouse betreibbar), kostenfreie private Repositories, einfach konfigurierbares CI/CD. Aber grundsätzlich: Geschmackssache.
            \item<3-> Branch-Anpassung
            \item<5-> Abfrage: Ist .git-credentials ein Problem? Hinweisen auf häufige Wiederverwendung von Mail-Passwort-Kombinationen
            \item<7-> \textbf{Diskutieren}: Passwort für key. Ist passwortloser key besser oder schlechter als .git-credentials? Hinweis auf keepass
            \item<8-> \textbf{Diskutieren}: Ist git push --force nach reset auf gepushten commit  eine gute Idee?
        \end{itemize}
    }
    \begin{itemize}
        \item Erstellung eines gitlab-Accounts
        \item<2-> Neues Projekt erstellen, README anpassen (online)
        \item<3-> clone, README anpassen, commit
        \item<4-> push... Aber wo geht es hin? git remote! Und was geht da hin? git branch
        \item<5-> Credentials merken...
        \item<6-> ...und gleich wieder löschen! 
        \item<7-> Alternative: \href{https://docs.gitlab.com/ee/ssh/}{SSH-Key}. Dafür: clone über SSH (oder remote anpassen?)
        \item<8-> Fehler rückgängig machen: git reset (vor und nach push)
    \end{itemize}
\end{frame}

\begin{frame}{Flutter}
    \note{
        \only<1-2>{
            Wichtig: Keine Werbung für Flutter, zeigt Herangehensweise an neues Projekt (selber keine Erfahrung mit Flutter)
            Beispiele:\\Java super für portable, performante Applikationen, aber mies für Prototyping\\Python super für Prototyping, aber schlecht für Stabilität\\PHP Totalausfall bei Wartbarkeit+Sicherheit, aber toll für schnelle kleine Webapp\\Wünsche JavaScript schnellen und schmerzhaften Tod, setze es trotzdem für interaktive Visualisierungen ein\\etc... \\Fazit: Keine silver bullet, man muss viele Stärken und Schwächen kennen, und z.T. Programmierkonzepte zwischen Sprachen übertragbar. Flexibilität ist wichtig!
        }
        \only<3->{
            Mini-Whiteboard: Was ist wichtig bei der Entscheidung für eine Sprache für ein Projekt?
            \textbf{Zeit: 5 Minuten}
            \begin{itemize}
                \item Entwickler: Einer? Mehrere? Community oder Spaltungen? Aktivität der Entwicklung? (wikipedia)
                \item Anzahl an reifen Projekten in Sprache (flutter.dev)
                \item Bekannte Schwachstellen und Reaktionen (exploit-db.com, cve.mitre.org etc), konkretes Beispiel: www.cvedetails.com, Oracle (Page 10), Java, Metasploit modules, CVE-2012-4681 Java 7 Applet Remote Code Execution. Google: Page 12. Dart: Dart TCP
                \item Geeignetheit der Features für Problem (bsp. R, shell, PHP)
                \item Verfügbarkeit von relevanten Bibliotheken (pub.dev, chart, flchart)
                \item Verfügbarkeit aktueller Entwicklungstools
                \item Lizenz: Vorhanden? Falls ja, Bedingungen, bsp. GPL vs. LGPL. BSD: Redist license!
            \end{itemize}
        }
    }
    Wichtig in Programmierung: Weniger eine konkrete Sprache, mehr breites Wissen und Verständnis von Konzepten. Also: Richtige Programmiersprache für das Problem wählen! \only<2->{Aber: Was heißt ``richtig?''}\only<4->{\textbf{ Und ist Flutter ``richtig''?}}
    \only<3->{
        \begin{itemize}
            \item[\only<4>{\textbullet}\only<5->{\checkmark}] Aktive Entwicklung
            \item[\only<4-5>{\textbullet}\only<6->{\checkmark}] Starke Aktuere in der Community
            \item[\only<4-6>{\textbullet}\only<7->{\checkmark}] Vorhandensein reifer Produkte
            \item[\only<4-7>{\textbullet}\only<8->{\checkmark}] Sicherheit
            \item[\only<4-8>{\textbullet}\only<9->{\checkmark}] Relevante Bibliotheken und Entwicklungstools
            \item[\only<4-9>{\textbullet}\only<10->{\checkmark}] Bekannte und nutzbare Lizenz
            \item[\only<4-10>{\textbullet}\only<11->{?}] Ausreichende ROI bei fehlenden Vorkenntnissen
        \end{itemize}
    }
\end{frame}

\begin{frame}{Live-Minimalbeispiel}
    \begin{itemize}
        \item Standardbeispiel Flutter: Startup-Namensgenerator
        \item<2-> ...ist das gut umgesetzt? Skaliert das?
    \end{itemize}
\end{frame}

\begin{frame}{MVC-Pattern}
    \note{
        \begin{itemize}
            \item Sehr altes Konzept, 1970er bei Xerox Palo Alto
            \item Popularität durch Web-Entwicklung, jetzt viele Frameworks, die MVC-Pattern implementieren
        \end{itemize}
        Vorteile:
        \begin{itemize}
            \item Einfachere Einführung von Änderungen, da Logik geclustert
            \item Einfachere Aufgabenteilung im Team (schlanke APIs)
            \item Umsetzung von GUIs für unterschiedliche Endgeräte mit gleichem MC
        \end{itemize}
        \textbf{Whiteboard-Aufgabe}: Was könnte ein Nachteil von MVC sein? \textbf{Zeit: 2 Minuten}
        \begin{itemize}
            \item Code-Duplikation, da teilweise logische Verschränkung
            \item Schwer konsistent zu entscheiden was wo hin (was Logik, was Control?)
            \item Boilerplate
        \end{itemize}
    }
    \begin{columns}
        \begin{column}{0.5\textwidth}
            \begin{itemize}
                \item Design-Pattern zur Aufgabentrennung
                \item Model: Daten und Business-Logik
                \item View: GUI
                \item Control: User-Interaktion (und eventuell View-Auswahl)
            \end{itemize}
        \end{column}
        \begin{column}{0.5\textwidth}
            \begin{figure}
                \centering
                \includegraphics[width=0.5\textwidth]{Bilder/MVC.png}
                \caption{Kommunikation im MVC-Pattern \cite{MVC}}
            \end{figure}
        \end{column}
    \end{columns}
\end{frame}


\begin{frame}{Live-Minimalbeispiel revisited}
    Übung: Und jetzt das Gleiche nochmal als MVC!
    \only<2->{
        \begin{itemize}
            \item Werden Model, View, Control alle benötigt?
            \item Eigene Entscheidung - aber:
            \begin{itemize}
                \item Warum wird was weggelassen?
                \item Skaliert das? Welche Probleme könnte das bringen?
            \end{itemize}
        \end{itemize}
    }
\end{frame}


\begin{frame}[allowframebreaks]{Quellenangaben}
    \printbibliography
\end{frame}

\begin{frame}{Lizenz}
    \begin{center}
        \includegraphics{Bilder/by.png}\\
        Alle Inhalte außer dem HTW-Logo, für die keine Quelle angegeben ist, sind eigenes Material und unter CC-BY 4.0\\
        \url{https://creativecommons.org/licenses/by/4.0/}\\
        lizenziert.\\\vspace{0.5cm}
        Sämtliche Nutzungs- und Verwertungsrechte für das Logo der HTW Berlin in allen hier verwendeten Formen liegen ausschließlich bei der HTW Berlin:\\
        \url{https://corporatedesign.htw-berlin.de/logos/logo-htw-berlin/}
    \end{center}
\end{frame}

\end{document}
