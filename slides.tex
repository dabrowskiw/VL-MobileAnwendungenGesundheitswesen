\documentclass[aspectratio=169,t]{beamer}
\usepackage[utf8]{inputenc}
\usepackage[T1]{fontenc}


\title{Mobile Anwendungen im Gesundheitswesen}
\date{WS 2019/2020}
\author[PWD]{Dr.-Ing. Piotr Wojciech Dabrowski}
\titlegraphic{Bilder/logo.png}
% https://pxhere.com/en/photo/1441931

\usepackage{HTWBeamerTemplate/beamerthemeHTW}
\subtitle{0: Allgemeines}
\addbibresource{Bilder/0/imagesources.bib}
\begin{document}

\setbeamertemplate{footline}[first]
\begin{frame}[noframenumbering]
\titlepage
\end{frame}

\setbeamertemplate{footline}[presentationbody] 

\begin{frame}{Vorstellung}
 \begin{itemize}
     \item<2-> Kurz zu mir
     \only<2-4>{
      \begin{itemize}
        \item<2-4> Kontakt: Piotr.Dabrowski@posteo.de - gerne nutzen!
        \item<3-4> Geboren 1981 in Warschau
        \item<3-4> Studium der Biotechnologie \& Informatik an der TU Berlin
        \item<3-4> Promotion über Auswertung von Hochdurchsatzdaten für Virus-Diagnostik
        \item<3-4> Aufbau der bioinformatischen Analytik für das NGS-Labor des RKI
        \item<3-4> Aufbau der Bioinformatics Core Facility am RKI
        \item<4> Hang zu unkonventionellen Vorlesungsmethoden - Feedback erwünscht!
      \end{itemize}
     }
     \note<4->{Erstes Mal Vorlesung, Background nicht in mobilen Applikationen\\
     Bitte um Geduld, Hinweise
     }
     \item<5-> Der Todesstern \& (ausgewählte) andere Hilfsmittel
     \item<9-> Sie \& Ihre Vorstellungen, Motivation und Vorkenntnisse
     \only<10->{
       \begin{itemize}
           \item Tafelbild - coming to a git repo near you starting tomorrow!
       \end{itemize}
     }
 \end{itemize}
 \only<5-8>{
  \begin{textblock}{15}(2,7)
   \includegraphics[width=3cm]{Bilder/0/Todesstern.jpg}
  \end{textblock}
 }
 \only<6-8>{
  \begin{textblock}{15}(5.5,8.25)
   \includegraphics[width=3cm]{Bilder/0/Tischnamensschild.png}
  \end{textblock}
 }
 \only<7-8>{
  \begin{textblock}{15}(9,7.5)
   \includegraphics[angle=77,origin=c,height=2cm]{Bilder/0/Tischnamensschild.png}
  \end{textblock}
 }
 \only<8>{
  \begin{textblock}{15}(11.5,8)
   \includegraphics[width=3cm]{Bilder/0/DSLR.png}
  \end{textblock}
 }
\end{frame}

\begin{frame}{Lernziel der Vorlesung}
 App-Entwicklung: Was kann man im Gesundheitswesen mit Apps machen?
\end{frame}

\begin{frame}{Inhalt der Vorlesung}
 Wie entwickelt man eine Gesundheitsapp? \uncover<2->{Am Beispiel eines Start-ups.}
\end{frame}

\begin{frame}{Inhalt der Übung}

\end{frame}

\begin{frame}{Benotung (?)}
 \note<2>{Muss nicht ``fertig werden'', Idee und Herangehensweise ausschlaggebend.}
 \begin{itemize}
     \item<1-> Klausur: $40\%$
     \item<1-> Übungsaufgaben: $40\%$
     \item<1-> Ergebnisvorstellung: $20\%$
 \end{itemize}
\end{frame}

\begin{frame}{Tests}
 \begin{itemize}
     \item Geschichte, TDD $\rightarrow$ BDD
     \item Vor- und Nachteile von Testherangehensweisen
     \item Warum ist Testen notwendig?
 \end{itemize}
\end{frame}

\begin{frame}{Flutter}
 \begin{itemize}
     \item Wichtig in Programmierung: Weniger konkrete Sprache, mehr breites Wissen
     \note{Beispiele:\\Java super für portable, performante Applikationen, aber mies für Prototyping\\Python super für Prototyping, aber schlecht für Stabilität\\PHP Totalausfall bei Wartbarkeit+Sicherheit, aber toll für schnelle kleine Webapp\\Wünsche JavaScript schnellen und schmerzhaften Tod, setze es trotzdem für interaktive Visualisierungen ein\\etc... \\Fazit: Keine silver bullet, man muss viele Stärken und Schwächen kennen, und z.T. Programmierkonzepte zwischen Sprachen übertragbar. Flexibilität ist wichtig!}
     \item Kurz Geschichte/Idee von Flutter
 \end{itemize}
\end{frame}

\begin{frame}{Rest}
 \begin{itemize}
    \item CI/CD/fullstack testing
    \item Empfehlung: Flutter, Jenkins. Keine Pflicht, andere Technologien können genutzt werden, wichtig ist das Ergebnis. Selber nicht viel Erfahrung mit mobile development, aber sehr viel mit Entwicklung allgemein. Wichtig: Flexibilität, technologien kommen und gehen, man muss flexibel bleiben und Tiefe aus einer Technologie in das schnelle Erlernen einer anderen transferieren können. Und: Je mehr Technologien mit Vor- und Nachteilen man kennt, um so besser kann man eine neue Technologie erlernen (Konzepte aus anderen erkennen) und ihre Schwächen und Stärken direkt einschätzen.
 \end{itemize}
\end{frame}

\begin{frame}{Live-Minimalbeispiel}
 \begin{itemize}
     \item EXIF-Daten aus Bildern in Galerie auslesen
     \item In flutter, mit code/integration/UI-Tests
     \item Live auf Handy (oder, falls nicht anzeigbar, Emulator)
 \end{itemize}
\end{frame}


\begin{frame}{Bildquellen}
\printbibliography
\end{frame}

\end{document}